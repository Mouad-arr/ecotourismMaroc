\documentclass[a4paper,12pt]{report}
\usepackage[utf8]{inputenc}
\usepackage[T1]{fontenc}
\usepackage{lmodern}
\usepackage{listings}
\renewcommand{\lstlistingname}{Listado}
\usepackage{float}
\usepackage{xcolor}

%\usepackage[symbols]{circuitikz}
\usepackage{amssymb}
\usepackage{array}
\usepackage{muStyleReport}
\usepackage[french]{babel}
\usepackage{mdframed}
\usepackage{array}

\usepackage{hyperref}                  % Liens hypertexte
\usepackage{geometry}                  % Gestion des marges
\geometry{margin=2cm}  
\usepackage{titlesec}

\usepackage{amsmath,amsfonts,amstext,amssymb,epsfig,bbm,wasysym}
\usepackage{gensymb}
%\usepackage{mathrsfs}
%\usepackage{eufrak}
%\usepackage{unicode-math}
%\setmathfont{STIX Math}
%\def\kay{\ensuremath{\mscrk}}

\usepackage{graphicx}

\usepackage{tabularx}
\usepackage{caption}
\usepackage{multicol}  

\definecolor{headerblue}{RGB}{41, 128, 185}
\definecolor{rowgray}{RGB}{240, 240, 240}


\usepackage{tikz}
\usepackage{pgfplots}
\usetikzlibrary{arrows,plotmarks}
\usepackage{circuitikz}
\usepackage{dirtree}
%\usetikzlibrary{fandigs}
\usetikzlibrary{positioning}
\usetikzlibrary{shapes,calc,decorations.markings,decorations.pathreplacing}
\usetikzlibrary{arrows,shapes,positioning}
\usetikzlibrary{decorations.markings,decorations.pathmorphing,decorations.pathreplacing}
\usetikzlibrary{calc,patterns,shapes.geometric}
\usetikzlibrary{trees, arrows.meta, shadows}

\newcommand{\titreTP}[1]{%
    \begin{tcolorbox}[colframe=black, colback=gray!10, width=\textwidth, sharp corners=south]
        \centering
        \LARGE \textbf{#1}
    \end{tcolorbox}
    \vspace{1cm} % Espacement après le titre
}
\newcommand{\definition}[2]{
    \begin{tcolorbox}[colframe=black, colback=gray!10, title=\textbf{Définition : #1}]
        #2
    \end{tcolorbox}
}

   \usepackage{tcolorbox}
%   \titleformat{\section}[block]
%  {\normalfont\Large\bfseries\color{red}}  % Style (police, taille, couleur)
%  {\thesection}{1em}{}
%  \titleformat{\subsection}[block]
%  {\normalfont\Large\bfseries\color{blue}}  % Style (police, taille, couleur)
%  {\thesection}{1em}{}
%   \titleformat{\chapter}[hang]{\bfseries\huge}{\thechapter.}{1em}{}


%\usepackage{tcolorbox}
%\usepackage{color,colortbl}
\begin{document}
\lstdefinelanguage{JavaScript}{
	morekeywords={typeof, new, true, false, catch, function, return, null, catch, switch, var, if, in, while, do, else, case, break, const, let},
	morecomment=[s]{/*}{*/},
	morecomment=[l]//,
	morestring=[b]",
	morestring=[b]'
}
% --- Page de garde ---
\begin{titlepage}
   \begin{minipage}{0.55\textwidth}
        \raggedright
       % \large
        \textbf{DLA-1}
    \end{minipage}
    \hfill
    \begin{minipage}{0.4\textwidth}
        \includegraphics[width=\textwidth]{ensa.png} % Chemin vers l'image
    \end{minipage}
    \vfill
    \begin{center}
   \huge Projet Web 2025
    \end{center}
   \vfill
   \begin{center}        
        \rule{\linewidth}{0.5mm}\\[0.4cm]
        {\Large \textbf{Titre : } Conception et réalisation  du site ÉcoTourisme Maroc }\\[0.5cm]
        {\large Technologie Web}\\[0.4cm]
        \rule{\linewidth}{0.5mm}\\[1.5cm]
  \end{center}
  \vfill
  \begin{center}
        \begin{minipage}[t]{0.4\textwidth}
            \begin{flushleft} \large
                \textbf{Préparé par :}\\
                Mouad Ouchelh\\
                Youssef Afella\\
                Achraf Boulhem
            \end{flushleft}
        \end{minipage}
        \hfill
        \begin{minipage}[t]{0.4\textwidth}
            \begin{flushright} \large
                \textbf{Encadré par :}\\
                Pr. Qazdar Aimad
            \end{flushright}
        \end{minipage}\\[2cm]
        \end{center}
        \vfill
        \begin{center}
        \large Année Universitaire : 2025/2026
    \end{center}
\end{titlepage}
\newpage
\begin{abstract}
Ce rapport présente la conception et le développement du site web \textbf{ÉcoTourisme Maroc}, une plateforme dédiée à la promotion du tourisme durable au Maroc. Le document s'articule autour d'une introduction contextualisant le projet, suivie d'un développement structuré en chapitres décrivant les différentes étapes : analyse des besoins et conception de l'architecture, développement des pages en HTML, mise en forme et design avec CSS, implémentation des fonctionnalités interactives en Javascript, et enfin le déploiement en ligne de la plateforme. Le site permet aux visiteurs d'explorer des destinations authentiques, de découvrir des activités éco-responsables et de consulter des articles de blog, tout en valorisant le patrimoine naturel et culturel marocain dans une démarche respectueuse de l'environnement. Le rapport se conclut par une évaluation critique du projet, identifiant ses points forts et ses limites, et propose des perspectives d'évolution pour enrichir l'expérience utilisateur et renforcer l'impact de la plateforme dans le domaine du tourisme responsable.
\end{abstract}
\tableofcontents
\newpage
\chapter{Introduction }
\section{Contexte}
Le tourisme figure parmi les secteurs clés de l'économie marocaine. Face aux menaces environnementales et à la nécessité d'un développement compatible avec la préservation des ressources naturelles, l'\textbf{écotourisme} se présente comme une alternative durable. Le projet \textit{ÉcoTourisme Maroc} consiste en la réalisation d'un site web vitrine destiné à promouvoir cette forme de tourisme au niveau national.
\section{Objectifs du projet }
\begin{itemize}
  \item Sensibiliser le public au tourisme durable et aux bonnes pratiques.
  \item Présenter des destinations, activités et hébergements responsables au Maroc.
  \item Fournir des conseils pratiques et des parcours de voyage éco‑responsables.
  \item Mettre en place une expérience utilisateur claire, responsive et visuelle.
\end{itemize}
\section{Périmètre}
Le site développé est un site statique (HTML/CSS/JS) hébergé sur GitHub Pages. Il ne comprend pas, dans sa version initiale, de back‑office ni de base de données. Le périmètre couvre : la page d'accueil, pages destinations, activités, blog/conseils, et page contact.

\section{Présentation du projet}
\subsection{Pourquoi un site de tourisme durable ?}
Le tourisme de masse génère des impacts environnementaux et sociaux considérables : dégradation des écosystèmes, forte empreinte carbone, pression sur les ressources locales et perte d'authenticité culturelle. Face à ces défis, le tourisme durable s'impose comme une alternative nécessaire, visant à minimiser ces impacts tout en valorisant les communautés locales.



Une plateforme web dédiée répond à plusieurs besoins essentiels : 
\begin{itemize}
\item Sensibiliser les voyageurs aux pratiques responsables
\item Centraliser les informations sur les destinations et activités écoresponsables
\item Promouvoir les acteurs locaux
\item  Faciliter la planification de séjours alignés avec des valeurs environnementales et éthiques.
\end{itemize}
\subsection{Pourquoi le Maroc ?}
Le Maroc présente des atouts exceptionnels pour le développement du tourisme durable. Le pays offre une grande diversité de paysages, du Sahara aux montagnes de l'Atlas, en passant par les côtes atlantiques et méditerranéennes.\\ Son patrimoine culturel millénaire, ses traditions berbères et ses médinas classées à l'UNESCO constituent un héritage précieux à préserver.


La biodiversité marocaine, riche en parcs nationaux et espèces endémiques, fait face à plusieurs menaces et nécessite une protection renforcée. Le tourisme responsable peut jouer un rôle essentiel dans cette préservation en valorisant les écosystèmes tout en soutenant les communautés locales. L'engagement du Maroc en faveur du développement durable, illustré par l'accueil de la COP22 et les récentes stratégies environnementales du pays, crée un contexte favorable pour encourager et développer ce type d'initiatives.


Avec plus de 13 millions de visiteurs annuels et une demande croissante pour des expériences authentiques, le Maroc dispose d'un potentiel important pour développer un modèle touristique durable qui bénéficie aux communautés locales tout en préservant son patrimoine naturel et culturel.
\section{Problématique}
Malgré la richesse naturelle et culturelle du Maroc, les informations relatives au tourisme durable restent dispersées et peu accessibles au grand public. Les voyageurs manquent souvent de repères clairs pour organiser des séjours respectueux de l’environnement et des communautés locales. Ce projet vise donc à répondre à ce manque à travers une plateforme web dédiée à l'écotourisme marocain.

%%%%%%%%%%%%%%%%%%%%%%%%%%%%%%%%%%%%%%%%%%%%%%%%%%%%%%%%%%%%%%%%%%%%%%%%%%%%%%%%%%%%

\chapter{Analyse des besoins}
Cette partie a pour objectif d’identifier les attentes des utilisateurs, les fonctionnalités nécessaires du site web ainsi que les contraintes techniques à respecter afin de concevoir une solution adaptée au contexte du tourisme écologique au Maroc.

\section{Analyse de l'existant}
Conformément aux exigences du projet , nous avons sélectionné trois plateformes similaires afin d'analyser leurs forces et d'identifier les meilleures pratiques en termes de contenu, de navigation et de design.

\subsection{Site 1 : Visit Morocco (Portail Officiel)}
\begin{itemize}
	\item \textbf{Contenu :} Informations exhaustives sur les régions et la culture marocaine.
	\item \textbf{Navigation :} Menu horizontal classique avec des sous-menus par destination.
	\item \textbf{Design :} Utilisation de photographies haute résolution mettant en avant les paysages.
	\begin{figure}[h!] 
		\centering
		\includegraphics[width=\textwidth]{visitmorocco.png}
		\caption{Visit Morocco}
		\label{fig:visitmorocco}
	\end{figure}
\end{itemize}

\subsection{Site 2 : Terre d'Aventure (Spécialiste randonnée)}
\begin{itemize}
	\item \textbf{Contenu :} Fiches techniques détaillées pour chaque circuit et engagement écologique.
	\item \textbf{Navigation :} Recherche avancée par niveau de difficulté et thématique.
	\item \textbf{Design :} Palette de couleurs axée sur le vert et le marron, rappelant la nature.
	\begin{figure}[h!] 
		\centering
		\includegraphics[width=\textwidth]{terdav.png}
		\caption{Terre d'Aventure}
		\label{fig:terdav}
	\end{figure}
\end{itemize}

\subsection{Site 3 : Ecoventura (Tourisme Responsable)}
\begin{itemize}
	\item \textbf{Contenu :} Focus sur l'impact environnemental et témoignages de voyageurs.
	\item \textbf{Navigation :} Structure simple et épurée favorisant l'expérience utilisateur.
	\item \textbf{Design :} Style minimaliste avec une typographie moderne et lisible.
	\begin{figure}[h!] 
		\centering
		\includegraphics[width=\textwidth]{ecoventura.png}
		\caption{Ecoventura}
		\label{fig:ecoventura}
	\end{figure}
\end{itemize}

\subsection{Synthèse et positionnement du projet}
L'analyse de ces sites montre l'importance d'un visuel fort et d'une navigation intuitive. Pour "Ecotourisme Maroc", nous retiendrons l'aspect informatif des sites officiels tout en y intégrant un formulaire interactif riche, nécessaire pour la validation des réservations via Javascript.

\section{Identification des utilisateurs}
Le site \textbf{ÉcoTourisme Maroc} s’adresse principalement aux touristes souhaitant découvrir le Maroc de manière responsable et durable. Les utilisateurs ciblés incluent :
\begin{itemize}
	\item Les touristes locaux et internationaux,
	\item Les amateurs de nature et d’écotourisme,
	\item Les voyageurs recherchant des informations sur les sites naturels marocains.
\end{itemize}

Ces utilisateurs disposent généralement de connaissances basiques en navigation web et recherchent une interface simple, claire et accessible.

\section{Besoins fonctionnels}
Les besoins fonctionnels représentent les différentes fonctionnalités que le site doit offrir. Les principaux besoins fonctionnels identifiés sont :
\begin{itemize}
	\item Affichage des destinations écotouristiques au Maroc,
	\item Présentation détaillée de chaque lieu (description, localisation, images),
	\item Navigation simple entre les différentes pages du site,
	\item Présence d’un formulaire de contact permettant aux visiteurs d’envoyer des messages,
	\item Validation des données saisies dans le formulaire à l’aide de Javascript.
\end{itemize}

\section{Besoins non fonctionnels}
En plus des fonctionnalités, le site doit respecter certains besoins non fonctionnels afin d’assurer une bonne expérience utilisateur :
\begin{itemize}
	\item Interface ergonomique et facile à utiliser,
	\item Design responsive adapté aux différents types d’écrans (ordinateurs, tablettes, smartphones),
	\item Temps de chargement optimisé,
	\item Compatibilité avec les navigateurs web modernes.
\end{itemize}

\section{Contraintes techniques}
Le développement du site \textbf{EcoTourisme} doit respecter les contraintes techniques imposées par le module \textbf{Conception de sites web}. Ces contraintes sont :
\begin{itemize}
	\item Utilisation du langage HTML5 pour la structure des pages,
	\item Utilisation de CSS3 pour la mise en forme et le design,
	\item Utilisation de Javascript pour l’interactivité et la validation des formulaires,
	\item Hébergement du site sur la plateforme GitHub Pages.
\end{itemize}

%%%%%%%%%%%%%%%%%%%%%%%%%%%%%%%%%%%%%%%%%%%%%%%%%%%%%%%%%%%%%%%%%%%%%%%%%%%%%%%%%%%%%%

\chapter{Conception de l'architecture}

\section{Architecture globale du site}
Le site web est structuré autour de plusieurs pages principales, chacune ayant un rôle précis dans la présentation et la valorisation du tourisme écologique au Maroc. Cette organisation permet d’offrir une navigation claire et cohérente, tout en facilitant l’accès à l’information pour les visiteurs.

\subsection{Page d’accueil}

La page d’accueil constitue le point d’entrée principal du site \textit{EcoTourisme}. Elle a pour rôle de présenter le concept général du site et de sensibiliser les visiteurs à l’importance de l’écotourisme au Maroc.

Cette page contient une introduction générale, des visuels représentatifs des paysages marocains ainsi que des liens permettant d’accéder rapidement aux différentes sections du site.

\subsection{Page Destinations}

La page \textit{Destinations} est dédiée à la présentation des principaux endroits écotouristiques à explorer au Maroc. Elle met en avant la diversité géographique du pays, incluant les montagnes, les déserts, les plages et les oasis.

Chaque destination est accompagnée d’une description détaillée, d’images illustratives et d’informations permettant aux visiteurs de mieux comprendre l’intérêt écologique et touristique du lieu.

\subsection{Page Activités}

La page \textit{Activités} présente les différentes activités écotouristiques que les visiteurs peuvent pratiquer au Maroc.

Cette page permet aux touristes de choisir des expériences adaptées à leurs préférences tout en respectant les principes du tourisme durable.

\subsection{Page Blog}

La page \textit{Blog} a pour objectif de partager des articles informatifs et éducatifs liés à l’écotourisme. Elle propose des contenus variés tels que des conseils de voyage, des récits d’expériences, ainsi que des articles de sensibilisation à la protection de l’environnement.

Cette section contribue à enrichir le contenu du site et à renforcer l’engagement des visiteurs.

\subsection{Page À propos}

La page \textit{À propos} présente le site \textit{EcoTourisme}, son objectif et sa vision. Elle explique la motivation derrière la création du projet ainsi que son engagement en faveur du développement durable et de la valorisation du patrimoine naturel marocain.

Cette page permet d’instaurer une relation de confiance avec les visiteurs.

\subsection{Page Contact}

La page \textit{Contact} permet aux visiteurs de communiquer avec les administrateurs du site. Elle contient un formulaire de contact permettant aux utilisateurs d’envoyer des messages ou des demandes d’information.

Les champs du formulaire sont validés à l’aide de Javascript afin d’assurer la fiabilité des données saisies.

\section{Charte Graphique}
La charte graphique de notre projet a été conçue pour refléter les valeurs de l'écotourisme : nature, sérénité et authenticité. Elle assure une cohérence visuelle sur l'ensemble des pages du site.

\subsection{Palette de Couleurs}
Nous avons opté pour des teintes organiques inspirées des paysages marocains, en privilégiant le contraste et la lisibilité :
\begin{itemize}
	\item \textbf{Vert Nature (\#21C45D) :} Utilisé pour le bouton d'appel à l'action (CTA) principal et les éléments actifs du menu. Il symbolise l'écologie et l'action.
	\item \textbf{Bleu Nuit Foncé (\#0A1929) :} Utilisé pour le pied de page (footer), offrant un ancrage visuel fort et professionnel.
	\item \textbf{Blanc Cassé / Fond Clair (\#F8FAFC) :} Couleur de fond pour garantir une interface claire et aérée.
\end{itemize}

\subsection{Typographie}
Le choix typographique s'est porté sur des polices sans-serif modernes pour assurer une lecture fluide sur tous les supports:
\begin{itemize}
	\item \textbf{Titres :} Utilisation d'une police grasse (Bold) pour hiérarchiser l'information et capter l'attention sur les slogans.
	\item \textbf{Corps de texte :} Typographie légère et épurée pour les descriptions, facilitant la lecture prolongée.
\end{itemize}

\subsection{Éléments d'Interface (UI)}
\begin{itemize}
	\item \textbf{Boutons :} Les boutons possèdent des angles arrondis pour un aspect accueillant et moderne.
	\item \textbf{Navigation :} Une barre de menu épurée avec des boutons distincts pour chaque section (Destinations, Activités, Blog, etc.), renforçant l'arborescence du site.
\end{itemize}

\section{Schéma de navigation}

La navigation entre les différentes pages du site est assurée par un menu de navigation principal présent sur l’ensemble des pages. Ce menu permet un accès rapide aux sections essentielles du site et améliore l’expérience utilisateur.

Un schéma de navigation a été défini afin de représenter les liens entre les différentes pages du site et garantir une structure cohérente.

\section{Choix technologiques}

Le choix des technologies s’est basé sur les besoins du projet et les objectifs pédagogiques du module. Le langage HTML5 a été utilisé pour structurer le contenu du site, tandis que CSS3 a permis de concevoir une interface visuelle attrayante et responsive. Javascript a été intégré afin d’ajouter de l’interactivité et d’assurer la validation des formulaires côté client.


%%%%%%%%%%%%%%%%%%%%%%%%%%%%%%%%%%%%%%%%%%%%%%%%%%%%%%%%%%%%%%%%%%%%%%%%%%%%%%%%%%%%%%

\chapter{Développement des pages en HTML }
Le développement des pages HTML du site ÉcoTourisme Maroc a été réalisé selon une approche progressive et structurée, s'appuyant sur les principes du développement web moderne et les standards internationaux du W3C (World Wide Web Consortium). Cette méthodologie consiste à concevoir d'abord la structure sémantique des pages avant d'intégrer les styles CSS et les interactions Javascript, en respectant le principe de séparation des préoccupations. L'objectif principal est d'obtenir un site web clair, accessible à tous les utilisateurs, facilement maintenable par les développeurs, et compatible avec l'ensemble des navigateurs modernes et supports (ordinateurs, tablettes, smartphones).

Cette approche itérative nous a permis de valider progressivement chaque composant avant d'ajouter des couches de complexité supplémentaires, garantissant ainsi la robustesse et la qualité du produit final.

\section{Choix des technologies}
Le projet repose sur un ensemble de technologies web standards, soigneusement sélectionnées pour assurer une large compatibilité, une facilité de déploiement et une pérennité du site.



\subsection{HTML5 - Structure sémantique}

Le langage HTML5 a été choisi comme base structurelle du site pour plusieurs raisons fondamentales :

\begin{itemize}
    \item \textbf{Sémantique enrichie :} HTML5 introduit des balises sémantiques (header, nav, main, section, article, aside, footer) qui donnent du sens au contenu et améliorent significativement le référencement naturel (SEO) ainsi que l'accessibilité pour les technologies d'assistance.
    
    \item \textbf{Compatibilité universelle :} Supporté par tous les navigateurs modernes (Chrome, Firefox, Safari, Edge) sans nécessiter de polyfills ou de bibliothèques supplémentaires.
    
    \item \textbf{Validation stricte :} Possibilité de valider le code selon les standards du W3C, garantissant une qualité et une conformité du code produit.
    
    \item \textbf{Multimédia natif :} Support intégré des éléments audio et vidéo sans dépendance à des plugins externes (comme Flash, désormais obsolète).
    
    \item \textbf{Formulaires avancés :} Nouveaux types d'inputs (email, tel, date, number) avec validation native côté navigateur.
\end{itemize}

% [FIGURE X.1 : Schéma des balises HTML5 sémantiques utilisées dans le projet]
% Suggestion : Créer un diagramme montrant la structure header > nav, main > section/article, footer
\tikzset{
  basicTag/.style={
    rectangle,
    rounded corners=3pt,
    draw=blue!40!black,
    very thick,
    minimum width=2.5cm,
    minimum height=1.2cm,
    align=center,
    font=\ttfamily\bfseries, 
    drop shadow,
  },
  containerTag/.style={
    basicTag,
    top color=orange!10,
    bottom color=orange!30,
  },
  innerTag/.style={
    basicTag,
    top color=green!10,
    bottom color=green!30,
    minimum width=2cm,
    font=\ttfamily\small
  },
  rootTag/.style={
    basicTag,
    top color=gray!10,
    bottom color=gray!30,
    minimum width=3cm
  },
  connecteur/.style={
    draw=blue!50!black,
    thick,
    -{Latex[length=3mm]}, 
    rounded corners=5pt
  }
}

\begin{tikzpicture}[
  level 1/.style={sibling distance=4.5cm, level distance=2.5cm},
  level 2/.style={sibling distance=2.5cm, level distance=2.5cm},
  edge from parent/.style={connecteur}, % Applique le style de flèche
  edge from parent path={(\tikzparentnode.south) -- ++(0,-0.5cm) -| (\tikzchildnode.north)} % Connecteurs à angle droit
]

\node[rootTag] {<body>}
  child { node[containerTag] {<header>}
    child { node[innerTag] {<nav>} }
  }
  child { node[containerTag] {<main>}
    child { node[innerTag] {<section>} }
    child { node[innerTag] {<article>} }
  }
  child { node[containerTag] {<footer>} };


\node[above=1cm, font=\bfseries\Large] at (current bounding box.north) {Structure Sémantique HTML5};

\end{tikzpicture}


\section{Organisation des fichiers et dossiers} 
L'architecture du projet suit une organisation modulaire et scalable, inspirée des meilleures pratiques de développement web. Cette structure facilite la navigation dans le code, la maintenance, et permet une collaboration efficace entre développeurs. 


\subsection{Arborescence complète du projet}

 \dirtree{%
.1 /eco-tourisme-maroc.
.2 index.html.
.2 destinations.html.
.2 apropos.html.
.2 Activite.html.
.2 consielsBlog.html.
.2 contact.html.
.2 article-responsable.html.
.2 article-ecolodges.html.
.2 hautAtalas.html.
.2 littoral.html.
.2 oasis\&dunes.html. % Notez le \ devant le &
.2 soutenirLesProjets.html.
.2 css/.
.3 AccueilStyle.css.
.3 ActiviteStyle.css.
.3 AproposStyle.css.
.3 base.css.
.3 BlogStyle.css.
.3 DestinationSty.css.
.3 navFooterSty.css.
.2 js/.
.3 contact.js.
.3 main.js.
.2 images/.
}
 
\subsection{Justification de l'organisation modulaire}

Cette structure présente plusieurs avantages majeurs pour le développement et la maintenance du projet :

\textbf{Séparation des préoccupations :}
Chaque type de ressource (HTML, CSS, Javascript, images) est regroupé dans son propre dossier dédié, facilitant la localisation rapide des fichiers et évitant les conflits de nommage.

\textbf{Modularité CSS :}
Les feuilles de style sont divisées par fonctionnalité et par page :
\begin{itemize}
    \item \texttt{base.css} : Contient le reset CSS, les variables globales, et les styles réutilisables
    \item \texttt{navFooterSty.css} : Composants communs à toutes les pages (en-tête, navigation, pied de page)
    \item Fichiers spécifiques par page : Permettent de charger uniquement les styles nécessaires
\end{itemize}

\textbf{Maintenabilité accrue :}
Les modifications d'un composant spécifique (par exemple, le style des cartes de destinations) peuvent être effectuées dans un fichier dédié sans risque d'effets de bord sur d'autres parties du site.

\textbf{Performance optimisée :}
La séparation des styles permet de ne charger que les ressources nécessaires pour chaque page, réduisant ainsi le temps de chargement initial.

\textbf{Scalabilité :}
L'ajout de nouvelles pages ou fonctionnalités suit simplement la convention de nommage établie, facilitant l'évolution future du site.

\textbf{Collaboration facilitée :}
Plusieurs développeurs peuvent travailler simultanément sur différentes parties du site (HTML, CSS, JS) sans conflits majeurs dans le système de contrôle de version Git.

Le tableau \ref{tab:fichiers_pages} présente la correspondance entre les pages HTML et leurs feuilles de style associées.

\begin{table}[h]
\centering
\caption{Pages HTML et leurs ressources CSS/JS associées}
\label{tab:fichiers_pages}
\begin{tabular}{|l|l|l|}
\hline
\textbf{Page HTML} & \textbf{CSS spécifique} & \textbf{JS spécifique} \\
\hline
index.html & AccueilStyle.css & main.js \\
destinations.html & DestinationSty.css & main.js \\
Activite.html & ActiviteStyle.css & main.js \\
apropos.html & AproposStyle.css & main.js \\
consielsBlog.html & BlogStyle.css & main.js \\
contact.html & - & contact.js + main.js \\
hautAtalas.html & DestinationSty.css & main.js \\
littoral.html & DestinationSty.css & main.js \\
oasis\&dunes.html & DestinationSty.css & main.js \\
\hline
\multicolumn{3}{|l|}{\textit{Note : base.css et navFooterSty.css sont chargés sur toutes les pages}} \\
\hline
\end{tabular}
\end{table}


\section{Développement de la structure HTML}
\subsubsection{Structure type d'une page}

Chaque page du site respecte la structure HTML5 suivante :

\begin{itemize}
    \item \texttt{<!DOCTYPE html>} : Déclaration du type de document HTML5
    \item \texttt{<html lang="fr">} : Élément racine avec indication de la langue française
    \item \texttt{<head>} : Métadonnées de la page (titre, description, styles, scripts)
    \item \texttt{<body>} : Contenu visible de la page, structuré ainsi :
    \begin{itemize}
        \item \texttt{<header>} : En-tête global du site (logo, navigation principale)
        \item \texttt{<main>} : Contenu principal unique de la page
        \item \texttt{<footer>} : Pied de page global (liens, informations de contact)
    \end{itemize}
\end{itemize}
Comme nous pouvons le voir sur la figure \ref{fig:structure_main}
 \begin{figure}[h!] 
  \centering
  \includegraphics[width=\textwidth]{exemple.png}
  

  \caption{Structure sémantique du code de la page principale}
  
  \label{fig:structure_main}
\end{figure}
\subsubsection{Balises sémantiques utilisées}

Le tableau \ref{tab:balises_semantiques} liste les principales balises sémantiques HTML5 utilisées dans le projet et leur rôle spécifique.

\begin{table}[h]
\centering
\caption{Balises HTML5 sémantiques utilisées}
\label{tab:balises_semantiques}
\begin{tabular}{|c|p{10cm}|}
\hline
\textbf{Balise} & \textbf{Utilisation dans le projet} \\
\hline
\texttt{<header>} & En-tête global du site contenant le logo et la navigation principale \\
\hline
\texttt{<nav>} & Menu de navigation (principal et secondaire) \\
\hline
\texttt{<main>} & Contenu principal unique de chaque page (une seule balise main par page) \\
\hline
\texttt{<section>}  & Regroupement thématique de contenu (ex: section destinations, section témoignages) \\
\hline
\texttt{<article>} & Contenu autonome réutilisable (ex: carte de destination, article de blog) \\
\hline
\texttt{<aside>} & Contenu complémentaire (ex: barre latérale avec conseils) \\
\hline
\texttt{<footer>} & Pied de page global avec liens utiles et informations légales \\
\hline
\texttt{<figure>} & Images avec légendes descriptives \\
\hline
\texttt{<figcaption>} & Légende associée à une image ou illustration \\
\hline
\end{tabular}
\end{table}
\section{Développement des pages principales}
Chaque page du site a été développée avec une attention particulière portée à son objectif spécifique, son public cible et l'expérience utilisateur qu'elle doit offrir. Cette section détaille la conception et le contenu des pages majeures du site.
\subsection{Page d'accueil (index.html)}

La page d'accueil constitue la vitrine du projet ÉcoTourisme Maroc. Elle présente immédiatement la philosophie du site à travers son slogan "Explorer • Respecter • Préserver" et offre une introduction visuelle aux richesses naturelles du Maroc.

\subsubsection{Objectifs de la page}

\begin{itemize}
    \item Communiquer immédiatement les valeurs du projet : exploration, respect et préservation
    \item Présenter visuellement la beauté naturelle du Maroc à travers des images immersives
    \item Orienter les visiteurs vers les destinations disponibles
    \item Créer une connexion émotionnelle avec les visiteurs à travers un message fort sur le tourisme responsable
\end{itemize}

\subsubsection{Structure et contenu réel de la page}

La page d'accueil adopte une structure minimaliste et impactante :

\textbf{1. En-tête et navigation :}
\begin{itemize}
    \item Logo et titre "ÉcoTourisme Maroc"
    \item Slogan principal : "Explorer • Respecter • Préserver"
    \item Menu de navigation horizontal avec 6 liens principaux :
    \begin{itemize}
        \item Accueil
        \item Destinations
        \item Activités
        \item Blog
        \item À propos
        \item Contact
    \end{itemize}
\end{itemize}

\textbf{2. Section Hero avec carrousel d'images :}
\begin{itemize}
    \item Carrousel automatique de 3 images panoramiques :
    \begin{itemize}
        \item nature1.jpg - Premier paysage naturel du Maroc
        \item nature7.jpg - Deuxième vue panoramique
        \item nature6.jpg - Troisième paysage écologique
    \end{itemize}
    \item Les images défilent automatiquement pour créer un effet immersif
    \item Texte alternatif descriptif : "Paysage écologique du Maroc"
\end{itemize}

\textbf{3. Section message principal :}
\begin{itemize}
    \item Titre accrocheur (h2) : "Explorez le Maroc autrement — entre nature, culture et respect."
    \item Paragraphe descriptif : "Découvrez un Maroc authentique où nature, traditions et cultures se rencontrent. Partez à l'aventure à travers des paysages uniques et des expériences locales conçues dans le respect de l'environnement et des communautés."
    \item Bouton d'appel à l'action : "Découvrir nos destinations" (lien vers destinations.html)
\end{itemize}

\textbf{4. Pied de page (Footer) :}
\begin{itemize}
    \item Copyright : "© 2025 ÉcoTourisme Maroc — Tous droits réservés."
    \item Lien rapide vers la page Contact
\end{itemize}

\subsubsection{Caractéristiques techniques}

\begin{itemize}
    \item \textbf{Carrousel d'images :} Implémenté en Javascript pour une transition fluide et automatique entre les 3 images du dossier \texttt{images/}
    \item \textbf{Design minimaliste :} Focalisation sur le visuel et le message sans surcharge d'informations
    \item \textbf{Hiérarchie claire :} Le message principal guide naturellement vers l'action (découvrir les destinations)
    \item \textbf{Responsive design :} Adaptation automatique du carrousel et du texte pour tous les appareils
\end{itemize}

 \begin{figure}[h!] 
  \centering
  \includegraphics[width=\textwidth]{home-page.png}
  

  \caption{Capture d'écran de la page d'accueil}
  
  \label{fig:structure_main}
\end{figure}
\subsection{Page Destinations (destinations.html)}
Point d'entrée principal vers les offres touristiques, cette page sert de carrefour de navigation.
\begin{itemize}
    \item \textbf{Objectif :} Présenter et comparer les trois écosystèmes (Montagne, Mer, Désert).
    \item \textbf{Structure :} Une grille de cartes ("Cards") comprenant pour chaque destination : une image représentative, un descriptif court et un lien vers la page détaillée.
\end{itemize}

\subsection{Pages de Détails par Destination}
Trois pages distinctes (\texttt{hautAtalas.html}, \texttt{littoral.html}, \texttt{oasis\&dunes.html}) suivent un gabarit commun pour assurer une cohérence visuelle et ergonomique :
\begin{enumerate}
    \item \textbf{En-tête immersif :} Image panoramique et introduction.
    \item \textbf{Contenu :} Présentation de la région, galerie photos et aspects écologiques.
    \item \textbf{Activités :} Liste des expériences spécifiques (Trekking, surf, bivouac).
    \item \textbf{Infos pratiques :} Saisonnalité et conseils voyageurs.
\end{enumerate}

\begin{table}[h]
\centering
\caption{Comparatif technique des destinations}
\label{tab:comparaison_destinations}
\renewcommand{\arraystretch}{1.2}
\begin{tabular}{|l|l|l|}
\hline
\textbf{Page} & \textbf{Thématique} & \textbf{Activités Clés} \\
\hline
\texttt{hautAtalas.html} & Montagne, Culture Berbère & Trekking, Randonnée \\
\hline
\texttt{littoral.html} & Côte, Biodiversité marine & Observation, Sports nautiques \\
\hline
\texttt{oasis\&dunes.html} & Désert, Culture Nomade & Bivouac, Chameau \\
\hline
\end{tabular}
\end{table}

\subsection{Page Activités (Activite.html)}
Cette page catalogue l'ensemble des offres sans distinction géographique, classées par typologie :
\begin{itemize}
    \item \textbf{Nature :} Randonnées, observation faune/flore.
    \item \textbf{Culture :} Rencontres locales, artisanat.
    \item \textbf{Aventure :} VTT, sports nautiques.
    \item \textbf{Écologie :} Reforestation, actions solidaires.
\end{itemize}
\textit{Design :} Utilisation de cartes avec visuels attractifs pour inciter à la découverte.

\subsection{Section Blog et Conseils (consielsBlog.html)}
Espace éditorial visant à améliorer le référencement (SEO) et à éduquer le visiteur.
\begin{itemize}
    \item \textbf{Page Index :} Liste les articles avec résumés et liens "Lire la suite".
    \item \textbf{Articles détaillés :} 
    \begin{itemize}
        \item \texttt{article-responsable.html} : Charte du voyageur éthique.
        \item \texttt{article-ecolodges.html} : Guide des hébergements durables.
    \end{itemize}
\end{itemize}

\subsection{Page À Propos (apropos.html)}
Page institutionnelle définissant l'identité du projet autour de la devise : \textbf{"Explorer • Respecter • Préserver"}. Elle explicite la mission de soutien aux communautés locales et l'engagement environnemental de l'équipe.

\subsection{Page Contact (contact.html)}
Interface de communication bidirectionnelle comprenant :
\begin{itemize}
    \item \textbf{Formulaire interactif :} Champs (Nom, Email, Sujet, Message) avec validation Javascript côté client.
    \item \textbf{Coordonnées :} Affichage direct des informations (email, réseaux sociaux).
    \item \textbf{Conformité :} Case à cocher pour le consentement RGPD.
\end{itemize}

\subsection{Page Soutenir les Projets (soutenirLesProjets.html)}
Page dédiée à l'engagement communautaire (Crowdfunding/Bénévolat). Elle présente les initiatives locales (ex: reforestation) et offre des mécanismes de transparence sur l'utilisation des fonds collectés.

\vspace{0.5cm}

\begin{table}[h]
\centering
\caption{Arborescence technique du site}
\label{tab:recapitulatif_pages}
\renewcommand{\arraystretch}{1.2}
\begin{tabularx}{\textwidth}{|l|l|X|}
\hline
\textbf{Page} & \textbf{Fichier HTML} & \textbf{Fonctionnalité Principale} \\
\hline
Accueil & \texttt{index.html} & Landing page, Navigation, CTA \\
\hline
Destinations & \texttt{destinations.html} & Hub vers les 3 régions \\
\hline
Détails & \texttt{*.html} (3 fichiers) & Infos spécifiques par région \\
\hline
Activités & \texttt{Activite.html} & Catalogue d'expériences \\
\hline
Blog & \texttt{consielsBlog.html} & Articles de fond et conseils \\
\hline
Contact & \texttt{contact.html} & Formulaire avec validation JS \\
\hline
À propos & \texttt{apropos.html} & Mission et valeurs \\
\hline
Soutien & \texttt{soutenirLesProjets.html} & Appel aux dons et bénévolat \\
\hline
\end{tabularx}
\end{table}

\section{Réutilisation des composants}

Les éléments communs tels que l'en-tête, le menu de navigation et le pied de page sont réutilisés sur l'ensemble des pages afin de garantir une cohérence visuelle et fonctionnelle à travers tout le site ÉcoTourisme Maroc.

\subsection{Composants globaux communs}

\subsubsection{En-tête et navigation}

L'en-tête du site est identique sur toutes les pages et comprend :

\begin{itemize}
    \item \textbf{Logo et titre} : "ÉcoTourisme Maroc" cliquable (retour à l'accueil)
    \item \textbf{Slogan} : "Explorer • Respecter • Préserver"
    \item \textbf{Menu de navigation} : 6 liens principaux (Accueil, Destinations, Activités, Blog, À propos, Contact)
    \item \textbf{Menu responsive} : Version mobile avec menu hamburger pour les petits écrans
\end{itemize}

Cette navigation cohérente permet aux visiteurs de se repérer facilement et d'accéder rapidement à n'importe quelle section du site depuis n'importe quelle page.

\begin{figure}[h!] 
  \includegraphics[width=0.5\textwidth]{mobile.png}
  \includegraphics[width=0.5\textwidth]{desktop.png}
  \caption{ Navigation desktop et mobile}
  
  \label{fig:structure_main}
\end{figure}

\subsubsection{Pied de page}

Le pied de page, également présent sur toutes les pages, contient :

\begin{itemize}
    \item \textbf{Copyright} : "© 2025 ÉcoTourisme Maroc — Tous droits réservés."
    \item \textbf{Lien rapide} : Accès direct à la page Contact
    \item \textbf{Design minimaliste} : Information essentielle sans surcharge
\end{itemize}

\subsection{Avantages de la réutilisation}

Cette approche de réutilisation des composants présente plusieurs bénéfices majeurs :

\begin{itemize}
    \item \textbf{Cohérence visuelle} : Les utilisateurs retrouvent les mêmes éléments sur chaque page, facilitant la navigation et renforçant l'identité du site
    \item \textbf{Maintenance simplifiée} : Une modification de l'en-tête ou du pied de page se répercute automatiquement sur toutes les pages en modifiant uniquement le fichier CSS correspondant (navFooterSty.css)
    \item \textbf{Développement accéléré} : La création de nouvelles pages est plus rapide grâce à la réutilisation de la structure HTML commune
    \item \textbf{Expérience utilisateur améliorée} : Navigation intuitive et prévisible sur l'ensemble du site
\end{itemize}

\begin{table}[h]
\centering
\caption{Composants réutilisés sur les pages du site}
\label{tab:composants_reutilises}
\begin{tabular}{|l|c|}
\hline
\textbf{Composant} & \textbf{Présent sur toutes les pages} \\
\hline
En-tête (Header) & \checkmark \\
\hline
Navigation principale & \checkmark \\
\hline
Pied de page (Footer) & \checkmark \\
\hline
Slogan "Explorer • Respecter • Préserver" & \checkmark \\
\hline
Liens de navigation (6 pages) & \checkmark \\
\hline
\end{tabular}
\end{table}

\section{Tests et validation}

Les pages développées ont été testées sur différents navigateurs et sur plusieurs tailles d'écran afin de garantir un affichage correct, une navigation fluide et une expérience utilisateur optimale sur tous les supports.

\subsection{Tests de compatibilité navigateurs}

Le site a été testé sur les navigateurs les plus utilisés pour assurer un rendu et un comportement identiques :

\begin{itemize}
    \item \textbf{Google Chrome} (version 120+) : Navigateur majoritaire
    \item \textbf{Mozilla Firefox} (version 121+) : Alternative open-source
    \item \textbf{Safari} (macOS/iOS) : Navigateur Apple
    \item \textbf{Microsoft Edge} : Navigateur Windows par défaut
\end{itemize}

Les points vérifiés incluent l'affichage correct des styles CSS, le fonctionnement du carrousel d'images, la validation des formulaires et la navigation responsive.

\subsection{Tests responsive}

Le site a été testé sur différentes tailles d'écran pour valider son adaptation automatique :

\textbf{Breakpoints testés :}
\begin{itemize}
    \item \textbf{Mobile} : 320px - 480px (smartphones)
    \item \textbf{Tablette} : 768px - 1024px (iPad, tablettes Android)
    \item \textbf{Desktop} : 1024px et plus (ordinateurs portables et fixes)
\end{itemize}

\textbf{Outils utilisés :}
\begin{itemize}
    \item Chrome DevTools (Device Mode)
    \item Tests sur appareils physiques (smartphone, tablette, ordinateur)
    \item Vérification du menu hamburger sur mobile
    \item Adaptation du carrousel d'images sur tous les formats
\end{itemize}

\subsection{Validation du code}

\subsubsection{Validation HTML5}

Le code HTML de toutes les pages a été validé avec le validateur W3C (https://validator.w3.org/) :
\begin{itemize}
    \item Vérification de la conformité HTML5
    \item Correction des erreurs de syntaxe
    \item Respect de la hiérarchie des balises sémantiques
\end{itemize}

\subsubsection{Validation CSS3}

Les feuilles de style ont été validées pour garantir leur conformité :
\begin{itemize}
    \item Validation via W3C CSS Validator
    \item Vérification de la compatibilité des propriétés CSS3
    \item Absence d'erreurs de syntaxe
\end{itemize}
\begin{figure}[h!] 
  \centering
  \includegraphics[width=\textwidth]{checking.png}
  \caption{Résultat de validation W3C}
  
  \label{fig:structure_main}
\end{figure}

\subsection{Tests fonctionnels}

\subsubsection{Navigation}
\begin{itemize}
    \item Vérification de tous les liens internes (navigation entre pages)
    \item Test du bouton "Découvrir nos destinations" sur la page d'accueil
    \item Fonctionnement correct des liens dans les cartes de destinations
    \item Navigation au clavier (accessibilité)
\end{itemize}

\subsubsection{Formulaire de contact}
\begin{itemize}
    \item Test de la validation des champs obligatoires
    \item Vérification du format email
    \item Affichage des messages d'erreur
    \item Test de soumission du formulaire
\end{itemize}

\subsubsection{Éléments interactifs}
\begin{itemize}
    \item Fonctionnement du carrousel d'images sur la page d'accueil
    \item Transitions automatiques entre les images
    \item Effets de survol sur les boutons et liens
    \item Menu hamburger sur mobile
\end{itemize}

\subsection{Tests de performance}

Des tests de performance ont été effectués pour garantir des temps de chargement rapides :

\begin{itemize}
    \item \textbf{Optimisation des images} : Compression et redimensionnement des photos
    \item \textbf{Temps de chargement} : Vérification que les pages se chargent en moins de 3 secondes
    \item \textbf{Nombre de requêtes} : Minimisation du nombre de fichiers externes
\end{itemize}

\subsection{Résultats des tests}

\begin{table}[h]
\centering
\caption{Récapitulatif des tests effectués}
\label{tab:resultats_tests}
\begin{tabular}{|l|c|c|}
\hline
\textbf{Type de test} & \textbf{Outils/Méthode} & \textbf{Résultat} \\
\hline
Compatibilité navigateurs & Chrome, Firefox, Safari, Edge & \checkmark Validé \\
\hline
Responsive design & DevTools, appareils physiques &\checkmark Validé \\
\hline
Validation HTML5 & W3C Validator & \checkmark Conforme \\
\hline
Validation CSS3 & W3C CSS Validator & \checkmark Conforme \\
\hline
Navigation et liens & Tests manuels & \checkmark Fonctionnel \\
\hline
Formulaire contact & Tests manuels & \checkmark Fonctionnel \\
\hline
Performance & Temps de chargement & \checkmark < 3 secondes \\
\hline
\end{tabular}
\end{table}

Tous les tests effectués confirment le bon fonctionnement du site sur l'ensemble des navigateurs et appareils testés, garantissant ainsi une expérience utilisateur optimale pour tous les visiteurs du site ÉcoTourisme Maroc.

\section{Conclusion du chapitre}

Le développement des pages HTML du site ÉcoTourisme Maroc a suivi une méthodologie rigoureuse et structurée, garantissant la qualité, l'accessibilité et la performance du produit final. L'utilisation de technologies web standards (HTML5, CSS3, Javascript) sans dépendance à des frameworks lourds assure la pérennité et la maintenabilité du site.

L'organisation modulaire des fichiers, la réutilisation systématique des composants, et le respect des standards W3C et WCAG 2.1 constituent les piliers d'un site professionnel et accessible à tous. Les tests exhaustifs effectués sur multiples navigateurs et appareils confirment la robustesse de l'architecture mise en place.

Le chapitre suivant détaillera la mise en forme CSS et les techniques de responsive design appliquées pour créer une expérience visuelle attractive et adaptative sur tous les supports.
\chapter{Mise en forme et design avec CSS}

La mise en forme du site ÉcoTourisme Maroc a été réalisée entièrement en CSS3, en adoptant une approche modulaire et responsive. L'objectif est de créer une identité visuelle cohérente qui reflète les valeurs écologiques du projet tout en garantissant une expérience utilisateur agréable sur tous les supports (ordinateurs, tablettes, smartphones).

\section{Architecture CSS du projet}

\subsection{Organisation des feuilles de style}

Les styles CSS sont organisés de manière modulaire dans le dossier \texttt{css/}, permettant une maintenance facilitée et une réutilisation optimale du code.

\subsubsection{Structure des fichiers CSS}

\begin{verbatim}
css/
 ├── base.css              # Styles de base et reset CSS
 ├── navFooterSty.css      # Navigation et pied de page
 ├── AccueilStyle.css      # Styles page d'accueil
 ├── DestinationSty.css    # Styles pages destinations
 ├── ActiviteStyle.css     # Styles page activités
 ├── AproposStyle.css      # Styles page à propos
 └── BlogStyle.css         # Styles blog et articles
\end{verbatim}

\textbf{Rôle de chaque fichier :}
\begin{itemize}
    \item \textbf{base.css} : Contient le reset CSS, les variables globales, la typographie de base et les classes utilitaires réutilisables
    \item \textbf{navFooterSty.css} : Gère l'apparence de l'en-tête, du menu de navigation et du pied de page (composants communs à toutes les pages)
    \item \textbf{Fichiers spécifiques} : Chaque page principale possède son propre fichier CSS pour les styles spécifiques, évitant ainsi la surcharge et permettant un chargement optimisé
\end{itemize}

\subsection{Approche de développement CSS}
\subsubsection{Principes de design appliqués}

\begin{itemize}
    \item \textbf{Simplicité} : Interface épurée mettant en avant le contenu visuel (paysages du Maroc)
    \item \textbf{Cohérence} : Uniformité des couleurs, typographies et espacements sur toutes les pages
    \item \textbf{Accessibilité} : Contrastes suffisants, tailles de texte lisibles, zones cliquables suffisamment grandes
    \item \textbf{Performance} : CSS optimisé, animations légères, absence de frameworks lourds
\end{itemize}
\section{Identité visuelle et charte graphique}

\subsection{Palette de couleurs}

La palette de couleurs a été choisie pour évoquer la nature et l'écologie, tout en assurant une bonne lisibilité et un contraste suffisant.

\textbf{Couleurs principales :}
\begin{itemize}
    \item \textbf{Vert naturel} : Couleur dominante évoquant l'écologie et la nature (utilisée pour les boutons CTA, liens actifs)
    \item \textbf{Blanc/Beige clair} : Arrière-plans pour la clarté et la lisibilité
    \item \textbf{Gris foncé/Noir} : Textes principaux pour un bon contraste
    \item \textbf{Tons terreux} : Marron, ocre pour rappeler les paysages marocains (désert, montagnes)
\end{itemize}
subsection{Typographie}

\textbf{Polices utilisées :}
\begin{itemize}
    \item \textbf{Titres} : Police sans-serif moderne et lisible (ex: Roboto, Open Sans, Montserrat)
    \item \textbf{Corps de texte} : Police sans-serif pour la clarté sur écran
    \item \textbf{Tailles} : 
    \begin{itemize}
        \item Titres h1 : 2.5rem (40px)
        \item Titres h2 : 2rem (32px)
        \item Titres h3 : 1.5rem (24px)
        \item Texte normal : 1rem (16px)
    \end{itemize}
\end{itemize}

\textbf{Hiérarchie typographique :}
Une hiérarchie claire est maintenue à travers le site pour faciliter la lecture et la compréhension de l'information.

\subsection{Espacements et grilles}

\begin{itemize}
    \item \textbf{Système d'espacement} : Basé sur des multiples de 8px (8px, 16px, 24px, 32px, 48px) pour une cohérence visuelle
    \item \textbf{Largeur maximale du contenu} : 1200px pour éviter des lignes de texte trop longues sur grands écrans
    \item \textbf{Marges et paddings} : Cohérents sur toutes les pages grâce aux variables CSS
\end{itemize}

\section{Styles des composants principaux}

\subsection{En-tête et navigation}

\subsubsection{Design de l'en-tête}

L'en-tête du site présente les caractéristiques suivantes :
\begin{itemize}
    \item \textbf{Position} : Fixe en haut de page (sticky header) pour un accès permanent à la navigation
    \item \textbf{Arrière-plan} : Blanc ou semi-transparent avec effet de flou lors du scroll
    \item \textbf{Logo} : Affiché à gauche avec le titre "ÉcoTourisme Maroc"
    \item \textbf{Slogan} : "Explorer • Respecter • Préserver" visible sous le logo ou dans le header
\end{itemize}

\subsubsection{Menu de navigation}

\textbf{Version desktop :}
\begin{itemize}
    \item Liste horizontale de liens alignés à droite
    \item Espacement uniforme entre les liens
    \item Effet de survol : changement de couleur, soulignement ou animation subtile
    \item Indication de la page active via un style différent (couleur, soulignement)
\end{itemize}

\textbf{Version mobile :}
\begin{itemize}
    \item Menu hamburger (icône trois barres) à droite
    \item Menu déroulant ou latéral au clic
    \item Animation d'ouverture/fermeture fluide
    \item Liens empilés verticalement pour faciliter le clic
\end{itemize}

subsection{Carrousel d'images (Page d'accueil)}

Le carrousel de la page d'accueil présente les caractéristiques suivantes :

\begin{itemize}
    \item \textbf{Taille} : Pleine largeur de l'écran, hauteur adaptative (70-100vh)
    \item \textbf{Transition} : Fondu enchaîné (fade) entre les 3 images (nature1.jpg, nature7.jpg, nature6.jpg)
    \item \textbf{Durée} : Changement automatique toutes les 5 secondes
    \item \textbf{Contrôles} : Éventuellement des points de navigation ou flèches pour navigation manuelle
    \item \textbf{Responsive} : Images adaptées automatiquement selon la taille d'écran (object-fit: cover)
\end{itemize}

\textbf{Styles CSS appliqués :}
\begin{itemize}
    \item Position relative pour contenir les images absolues
    \item Animations CSS (fadeIn/fadeOut) ou Javascript pour les transitions
    \item Optimisation des images pour performance (lazy loading)
\end{itemize}

\begin{figure}[h!] 
  \includegraphics[width=\textwidth]{transition.png}
  \caption{  Carrousel d'images avec transitions}
  
  \label{fig:structure_main}
\end{figure}


\subsection{Boutons et appels à l'action}

Les boutons CTA (Call To Action) sont stylisés de manière cohérente :

\textbf{Bouton principal :}
\begin{itemize}
    \item Couleur de fond verte (rappel écologie)
    \item Texte blanc en gras
    \item Bordures arrondies (border-radius: 5-10px)
    \item Padding généreux pour faciliter le clic
    \item Effet de survol : assombrissement de la couleur, légère élévation (box-shadow)
    \item Transition douce sur toutes les propriétés
\end{itemize}

\textbf{Exemple de bouton :} "Découvrir nos destinations" sur la page d'accueil

\subsection{Cartes de destinations}

Les cartes présentant les destinations suivent un design uniforme :

\begin{itemize}
    \item \textbf{Structure} : Image en haut, texte en bas
    \item \textbf{Bordures} : Ombrage léger (box-shadow) pour effet de profondeur
    \item \textbf{Espacement} : Marges cohérentes entre les cartes
    \item \textbf{Effet de survol} : Élévation de la carte (augmentation du box-shadow), zoom léger de l'image
    \item \textbf{Image} : Ratio 16:9 ou 4:3, couvrant toute la largeur de la carte
    \item \textbf{Texte} : Titre, description courte, lien "En savoir plus"
\end{itemize}

\subsection{Pied de page}

Le footer présente un design minimaliste :

\begin{itemize}
    \item Arrière-plan gris clair ou vert foncé
    \item Texte centré avec copyright
    \item Lien vers la page Contact
    \item Padding suffisant pour aérer le contenu
    \item Séparation visuelle claire avec le contenu principal (ligne ou espacement)
\end{itemize}

\section{Responsive Design}
\subsection{Techniques CSS utilisées}

\subsubsection{Flexbox}

Utilisé pour :
\begin{itemize}
    \item Alignement du header (logo à gauche, navigation à droite)
    \item Organisation du footer
    \item Centrage vertical et horizontal des éléments
\end{itemize}

\subsubsection{CSS Grid}

Utilisé pour :
\begin{itemize}
    \item Grille de cartes de destinations (3 colonnes desktop, 2 tablette, 1 mobile)
    \item Layout général de certaines pages
    \item Organisation des sections avec plusieurs colonnes
\end{itemize}

\subsubsection{Media Queries}

Permettent l'adaptation automatique du design selon la taille d'écran en modifiant :
\begin{itemize}
    \item Le nombre de colonnes dans les grilles
    \item Les tailles de police
    \item Les espacements et marges
    \item L'affichage ou masquage de certains éléments
    \item La disposition des éléments (flex-direction: column sur mobile)
\end{itemize}

\section{Animations et transitions}

\subsection{Animations CSS}

Des animations subtiles améliorent l'expérience utilisateur sans alourdir le site :

\textbf{Carrousel d'images :}
\begin{itemize}
    \item Transition en fondu (fade) entre les images
    \item Durée : 1 seconde
    \item Fonction d'accélération : ease-in-out
\end{itemize}

\textbf{Effets de survol :}
\begin{itemize}
    \item Liens : Changement de couleur progressif (transition: 0.3s)
    \item Boutons : Changement de couleur + élévation (box-shadow)
    \item Cartes : Élévation et zoom léger de l'image (transform: scale(1.05))
\end{itemize}

\textbf{Menu mobile :}
\begin{itemize}
    \item Animation d'ouverture/fermeture du menu hamburger
    \item Transformation de l'icône hamburger en X
    \item Transition fluide des liens (slide-in)
\end{itemize}

\subsection{Optimisation des performances}

Les animations sont optimisées pour la performance :
\begin{itemize}
    \item Utilisation de \texttt{transform} et \texttt{opacity} plutôt que de propriétés déclenchant des reflows
    \item Durées courtes (0.3s à 0.5s) pour éviter la lenteur
    \item Fonction \texttt{will-change} pour les animations fréquentes
    \item Désactivation des animations sur les appareils à faible performance (prefers-reduced-motion)
\end{itemize}


\section{Optimisation et bonnes pratiques}

\subsection{Performance CSS}

\subsubsection{Minification}

En production, les fichiers CSS sont :
\begin{itemize}
    \item Minifiés pour réduire la taille (suppression des espaces, commentaires)
    \item Potentiellement concaténés en un seul fichier pour réduire les requêtes HTTP
\end{itemize}

\subsubsection{Chargement optimisé}

\begin{itemize}
    \item \textbf{base.css et navFooterSty.css} : Chargés sur toutes les pages (styles communs)
    \item \textbf{Styles spécifiques} : Chargés uniquement sur les pages concernées (ex: AccueilStyle.css uniquement sur index.html)
    \item \textbf{Fonts} : Chargement optimisé des polices Google Fonts (font-display: swap)
\end{itemize}

\subsection{Compatibilité navigateurs}

\subsubsection{Préfixes vendeurs}

Utilisation de préfixes pour assurer la compatibilité avec les navigateurs plus anciens :
\begin{verbatim}
.element {
    -webkit-transition: all 0.3s ease;
    -moz-transition: all 0.3s ease;
    transition: all 0.3s ease;
}
\end{verbatim}

\subsubsection{Fallbacks}

Des solutions de repli sont prévues pour les fonctionnalités CSS3 avancées non supportées par certains navigateurs.

\subsection{Accessibilité}

\textbf{Contrastes :}
\begin{itemize}
    \item Ratio minimum de 4.5:1 entre texte et arrière-plan (WCAG AA)
    \item Texte blanc sur fond vert vérifié pour le contraste
\end{itemize}

\textbf{Focus visible :}
\begin{itemize}
    \item Outline visible lors de la navigation au clavier
    \item Style de focus personnalisé pour les boutons et liens
\end{itemize}

\textbf{Tailles de clic :}
\begin{itemize}
    \item Boutons et liens d'au moins 44x44 pixels sur mobile (recommandation WCAG)
    \item Espacement suffisant entre les éléments cliquables
\end{itemize}

\section{Variables CSS et maintenabilité}

\subsection{Utilisation des custom properties}

Les variables CSS sont définies dans \texttt{base.css} pour faciliter les modifications globales :

\begin{verbatim}
:root {
    --bg: #f7fbff;
    --primary: #1a2e37; 
    --accent: #1ab95a;  
    --muted: #7a8791;
    --card: #ffffff;
    --glass: rgba(255,255,255,0.7);
    --radius: 12px;
  --max-width: 1200px;
  --footer-height: 72px; 
}
\end{verbatim}

\subsection{Avantages des variables}

\begin{itemize}
    \item \textbf{Cohérence} : Les mêmes valeurs utilisées partout
    \item \textbf{Maintenance} : Modification globale en changeant une seule valeur
    \item \textbf{Thématisation} : Possibilité de créer des thèmes alternatifs facilement
    \item \textbf{Lisibilité} : Code CSS plus compréhensible avec des noms de variables explicites
\end{itemize}

\begin{table}[h]
\centering
\caption{Récapitulatif des techniques CSS utilisées}
\label{tab:techniques_css}
\begin{tabular}{|l|p{10cm}|}
\hline
\textbf{Technique} & \textbf{Utilisation dans le projet} \\
\hline
Flexbox & Alignement header/footer, centrage d'éléments \\
\hline
CSS Grid & Grille de cartes destinations, layouts multi-colonnes \\
\hline
Media Queries & Responsive design (mobile, tablette, desktop) \\
\hline
Animations & Carrousel, effets de survol, menu mobile \\
\hline
Variables CSS & Couleurs, espacements, typographies \\
\hline
Transitions & Effets de survol fluides (0.3s ease) \\
\hline
Box-shadow & Profondeur des cartes, élévation au survol \\
\hline
Transform & Zoom images, animations légères \\
\hline
\end{tabular}
\end{table}

\section{Conclusion du chapitre}

La mise en forme CSS du site ÉcoTourisme Maroc a été réalisée avec une attention particulière portée à la cohérence visuelle, la performance et l'accessibilité. L'approche modulaire adoptée facilite la maintenance et l'évolution future du site, tandis que le responsive design garantit une expérience optimale sur tous les appareils.

L'identité visuelle, basée sur des couleurs naturelles et une typographie claire, reflète les valeurs écologiques du projet. Les animations subtiles et les interactions fluides améliorent l'engagement des utilisateurs sans compromettre les performances.

Le chapitre suivant abordera l'ajout d'interactivité avec Javascript pour enrichir l'expérience utilisateur.

\chapter{Implémentation des fonctionnalités interactives en Javascript}

Javascript est le langage de programmation qui permet de rendre un site web interactif et dynamique. Dans ce chapitre, nous présentons les fonctionnalités Javascript que nous avons développées pour le site ÉcoTourisme Maroc. En tant que débutants en développement web, nous avons créé deux scripts principaux : \texttt{main.js} pour les interactions générales du site et \texttt{contact.js} pour la validation du formulaire de contact.

\section{Introduction à Javascript dans notre projet}

Javascript est un langage de programmation côté client qui s'exécute directement dans le navigateur de l'utilisateur. Il permet de :
\begin{itemize}
	\item Réagir aux actions de l'utilisateur (clics, survol, saisie)
	\item Modifier dynamiquement le contenu des pages
	\item Valider les formulaires avant envoi
	\item Créer des animations et effets visuels
	\item Améliorer l'expérience utilisateur globale
\end{itemize}

\subsection{Organisation des fichiers Javascript}

Notre projet contient deux fichiers Javascript principaux dans le dossier \texttt{scripts/} :

\begin{verbatim}
	scripts/
	├─ main.js          # Interactions générales du site
	└── contact.js       # Validation du formulaire de contact
\end{verbatim}

Ces fichiers sont chargés dans les pages HTML avec la balise \texttt{<script>} :

\begin{verbatim}
	<script src="scripts/main.js"></script>
	<script src="scripts/contact.js"></script>
\end{verbatim}

% [IMAGE: Capture d'écran de la structure des dossiers du projet]

\section{Script principal : main.js}

Le fichier \texttt{main.js} contient toutes les fonctionnalités Javascript communes à l'ensemble du site. Nous avons utilisé une technique appelée IIFE (Immediately Invoked Function Expression) pour encapsuler notre code et éviter les conflits avec d'autres scripts.

\subsection{Structure globale du script}

Notre script principal est organisé comme suit :
\begin{figure}[H] 
	\centering
	\includegraphics[width=\textwidth]{js1.png}
	\caption{main.js}
	
	\label{fig:structure_main}
\end{figure}

\subsection{Fonctions utilitaires}

Nous avons créé deux fonctions utilitaires pour simplifier la sélection d'éléments HTML :

\begin{lstlisting}[language=JavaScript, caption=Fonctions utilitaires de sélection]
	function qs(sel, el=document){ 
		return el.querySelector(sel) 
	}
	
	function qsa(sel, el=document){ 
		return Array.from(el.querySelectorAll(sel)) 
	}
\end{lstlisting}

\textbf{Explication :}
\begin{itemize}
	\item \texttt{qs} : Raccourci pour \texttt{querySelector} - sélectionne UN élément
	\item \texttt{qsa} : Raccourci pour \texttt{querySelectorAll} - sélectionne TOUS les éléments correspondants
	\item \texttt{el=document} : Paramètre par défaut, cherche dans tout le document si non spécifié
	\item \texttt{Array.from()} : Convertit la liste d'éléments en tableau pour faciliter les manipulations
\end{itemize}

Ces fonctions nous permettent d'écrire du code plus court et lisible.

% [IMAGE: Schéma expliquant querySelector vs querySelectorAll]

\section{Menu de navigation mobile}

L'une des fonctionnalités principales de notre site est le menu mobile responsive. Sur les petits écrans, le menu de navigation se transforme en menu hamburger.

\subsection{Principe de fonctionnement}

Le menu mobile fonctionne selon le principe suivant :
\begin{enumerate}
	\item Un bouton hamburger (trois barres) est visible sur mobile
	\item Au clic sur ce bouton, le menu s'ouvre ou se ferme
	\item L'utilisateur peut aussi fermer le menu en appuyant sur la touche Échap
	\item L'attribut ARIA \texttt{aria-expanded} indique l'état ouvert/fermé pour l'accessibilité
\end{enumerate}

\subsection{Code du menu mobile}

Voici le code complet que nous avons écrit :

\begin{figure}[H] 
	\centering
	\includegraphics[width=\textwidth]{js2.png}
	\caption{main.js / initMobileNav()}
	
	\label{fig:structure_main}
\end{figure}

\subsection{Explication détaillée}

\textbf{Étape 1 : Sélection des éléments}
\begin{lstlisting}[language=JavaScript]
	const toggle = qs('.nav-toggle');
	const header = qs('header');
\end{lstlisting}
On récupère le bouton hamburger (classe \texttt{.nav-toggle}) et l'élément \texttt{<header>}.

\textbf{Étape 2 : Vérification de l'existence}
\begin{lstlisting}[language=JavaScript]
	if(!toggle || !header) return;
\end{lstlisting}
Si un élément n'existe pas sur la page, on arrête la fonction pour éviter les erreurs.

\textbf{Étape 3 : Gestion du clic}
\begin{lstlisting}[language=JavaScript]
	toggle.addEventListener('click', ()=>{ /* ... */ });
\end{lstlisting}
On écoute les clics sur le bouton hamburger.

\textbf{Étape 4 : Lecture de l'état actuel}
\begin{lstlisting}[language=JavaScript]
	const expanded = toggle.getAttribute('aria-expanded') === 'true';
\end{lstlisting}
On vérifie si le menu est déjà ouvert en lisant l'attribut \texttt{aria-expanded}.

\textbf{Étape 5 : Inversion de l'état}
\begin{lstlisting}[language=JavaScript]
	toggle.setAttribute('aria-expanded', String(!expanded));
	header.classList.toggle('nav-open');
\end{lstlisting}
On inverse l'état : si ouvert, on ferme ; si fermé, on ouvre.

\textbf{Étape 6 : Fermeture au clavier}
\begin{lstlisting}[language=JavaScript]
	document.addEventListener('keydown', (e)=>{
		if(e.key === 'Escape' && header.classList.contains('nav-open')){
			// Fermer le menu
		}
	});
\end{lstlisting}
On permet à l'utilisateur de fermer le menu en appuyant sur Échap.

% [IMAGE: Capture d'écran du menu mobile ouvert et fermé]

\subsection{Le CSS correspondant}

Le Javascript ajoute/retire simplement la classe \texttt{nav-open}. C'est le CSS qui gère l'apparence :

\begin{verbatim}
	/* Menu cache par defaut sur mobile */
	header nav {
		display: none;
	}
	
	/* Menu visible quand la classe nav-open est presente */
	header.nav-open nav {
		display: block;
	}
\end{verbatim}

\section{Animation du slider d'images}

Notre site possède un slider (carrousel) d'images qui défile automatiquement sur la page d'accueil. Nous avons ajouté une fonctionnalité pour mettre en pause l'animation quand l'utilisateur survole le slider avec sa souris.

\subsection{Principe du slider}

Le slider fonctionne ainsi :
\begin{itemize}
	\item Les images défilent automatiquement grâce à une animation CSS
	\item Quand l'utilisateur survole le slider, l'animation se met en pause
	\item Quand l'utilisateur enlève sa souris, l'animation reprend
\end{itemize}

\subsection{Code de la pause du slider}

\begin{figure}[H] 
	\centering
	\includegraphics[width=\textwidth]{js3.png}
	\caption{main.js / initSliderPause()}
	
	\label{fig:structure_main}
\end{figure}

\subsection{Explication du code}

\begin{enumerate}
	\item \textbf{Sélection} : On récupère le conteneur \texttt{.slider} et l'élément \texttt{.slide}
	
	\item \textbf{mouseenter} : Événement déclenché quand la souris entre dans la zone du slider
	\begin{itemize}
		\item On ajoute la classe \texttt{paused} qui arrête l'animation CSS
	\end{itemize}
	
	\item \textbf{mouseleave} : Événement déclenché quand la souris sort de la zone
	\begin{itemize}
		\item On retire la classe \texttt{paused}, l'animation reprend
	\end{itemize}
\end{enumerate}

\subsection{Animation CSS correspondante}

L'animation est définie en CSS :

\begin{verbatim}
	.slide {
		animation: slideAnimation 15s infinite;
	}
	
	.slide.paused {
		animation-play-state: paused;
	}
	
	@keyframes slideAnimation {
		0%, 100% { transform: translateX(0); }
		33% { transform: translateX(-100%); }
		66% { transform: translateX(-200%); }
	}
\end{verbatim}

% [IMAGE: Illustration du slider avec images qui défilent]

\section{Marquage du lien actif dans la navigation}

Pour améliorer l'expérience utilisateur, nous avons créé une fonction qui met automatiquement en évidence le lien de navigation correspondant à la page actuelle.

\subsection{Objectif}

Si l'utilisateur est sur la page \texttt{destinations.html}, le lien "Destinations" dans le menu doit avoir un style différent (couleur, soulignement, etc.) pour indiquer qu'il s'agit de la page actuelle.

\subsection{Code de marquage actif}


\begin{figure}[H] 
	\centering
	\includegraphics[width=\textwidth]{js4.png}
	\caption{main.js / markActiveNav()}
	
	\label{fig:structure_main}
\end{figure}

\subsection{Explication pas à pas}

\textbf{1. Récupération du nom de la page :}
\begin{lstlisting}[language=JavaScript]
	const path = location.pathname.split('/').pop() || 'index.html';
\end{lstlisting}
\begin{itemize}
	\item \texttt{location.pathname} : Donne le chemin de l'URL actuelle
	\item \texttt{split('/')} : Découpe le chemin en morceaux
	\item \texttt{.pop()} : Prend le dernier morceau (le nom du fichier)
	\item \texttt{|| 'index.html'} : Si vide, utilise 'index.html' par défaut
\end{itemize}

Exemple : Si l'URL est \texttt{https://site.com/pages/contact.html}, \texttt{path} vaudra \texttt{contact.html}

\textbf{2. Parcours des liens :}
\begin{lstlisting}[language=JavaScript]
	qsa('nav.main-nav a').forEach(a=>{ /* ... */ })
\end{lstlisting}
On sélectionne tous les liens (\texttt{<a>}) dans la navigation et on les parcourt un par un.

\textbf{3. Comparaison et marquage :}
\begin{lstlisting}[language=JavaScript]
	if(path === href || 
	(href.endsWith('index.html') && path === '')){
		a.classList.add('active');
	}
\end{lstlisting}
Si le \texttt{href} du lien correspond à la page actuelle, on ajoute la classe \texttt{active}.

% [IMAGE: Menu de navigation avec lien actif mis en évidence]

\section{Bouton de retour en haut de page}

Pour améliorer la navigation sur les pages longues, nous avons créé un bouton "Retour en haut" qui apparaît automatiquement quand l'utilisateur descend dans la page.

\subsection{Fonctionnement}

\begin{itemize}
	\item Le bouton est caché par défaut
	\item Il apparaît quand l'utilisateur a scrollé plus de 200 pixels vers le bas
	\item Au clic, la page remonte en haut avec une animation fluide
	\item Le bouton disparaît quand l'utilisateur est en haut de page
\end{itemize}

\subsection{Code du bouton retour en haut}

\begin{figure}[H] 
	\centering
	\includegraphics[width=\textwidth]{js5.png}
	\caption{main.js / createBackToTop()}
	
	\label{fig:structure_main}
\end{figure}

\subsection{Explication détaillée}

\textbf{Création du bouton :}
\begin{lstlisting}[language=JavaScript]
	const btn = document.createElement('button');
	btn.className = 'back-to-top';
	btn.innerText = '\u2191';
\end{lstlisting}
On crée un nouvel élément \texttt{<button>} en Javascript, on lui donne une classe CSS et on ajoute une flèche ↑.

\textbf{Ajout à la page :}
\begin{lstlisting}[language=JavaScript]
	document.body.appendChild(btn);
\end{lstlisting}
On insère le bouton à la fin du \texttt{<body>}.

\textbf{Action au clic :}
\begin{lstlisting}[language=JavaScript]
	btn.addEventListener('click', ()=> {
		window.scrollTo({top:0, behavior:'smooth'});
	});
\end{lstlisting}
Quand on clique, \texttt{window.scrollTo()} fait remonter la page en haut (\texttt{top:0}) avec une animation fluide (\texttt{behavior:'smooth'}).

\textbf{Affichage conditionnel :}
\begin{lstlisting}[language=JavaScript]
	window.addEventListener('scroll', ()=>{
		if(window.scrollY > 200) {
			btn.style.display = 'block';
		} else {
			btn.style.display = 'none';
		}
	})
\end{lstlisting}
On écoute l'événement \texttt{scroll}. Si \texttt{window.scrollY} (position de scroll verticale) dépasse 200 pixels, on affiche le bouton, sinon on le cache.

% [IMAGE: Bouton retour en haut visible en bas à droite de la page]

\section{Initialisation au chargement de la page}

Toutes nos fonctions sont appelées quand la page est complètement chargée :

\begin{figure}[H] 
	\centering
	\includegraphics[width=0.7\textwidth]{js6.png}
	\caption{}
	
	\label{fig:structure_main}
\end{figure}

\textbf{Pourquoi DOMContentLoaded ?}

L'événement \texttt{DOMContentLoaded} est déclenché quand tout le HTML est chargé et que le DOM (Document Object Model) est prêt à être manipulé. C'est important car si on essaie de sélectionner des éléments avant qu'ils existent, le code ne fonctionnera pas.

\section{Validation du formulaire de contact}

Le deuxième fichier Javascript \texttt{contact.js} gère entièrement la validation du formulaire de contact. C'est le script le plus complexe de notre projet car il vérifie de nombreux champs différents.

\subsection{Structure du formulaire HTML}

Notre formulaire de contact contient les champs suivants :

\begin{verbatim}
	<form id="contactForm">
	<input id="fullname" type="text" placeholder="Nom complet">
	<input id="age" type="number" placeholder="Age">
	<input id="email" type="email" placeholder="Email">
	<input id="password" type="password" placeholder="Mot de passe">
	<input id="confirmPassword" type="password" 
	placeholder="Confirmer le mot de passe">
	<input id="fileInput" type="file">
	<textarea id="message" maxlength="500" 
	placeholder="Votre message"></textarea>
	<button type="submit">Envoyer</button>
	</form>
\end{verbatim}

% [IMAGE: Capture d'écran du formulaire de contact]

\subsection{Initialisation des variables}

Au début du script, on sélectionne tous les éléments du formulaire :

\begin{figure}[H] 
	\centering
	\includegraphics[width=0.9\textwidth]{js7.png}
	\caption{}
	
	\label{fig:structure_main}
\end{figure}

\textbf{Explication :}
\begin{itemize}
	\item On utilise \texttt{getElementById()} pour récupérer chaque champ par son ID
	\item \texttt{allowedExt} : tableau des extensions de fichiers autorisées
	\item Tout le code est dans un \texttt{DOMContentLoaded} pour s'assurer que les éléments existent
\end{itemize}

\subsection{Fonctions utilitaires d'affichage des erreurs}

Nous avons créé des fonctions pour afficher et effacer les messages d'erreur :

\begin{figure}[H] 
	\centering
	\includegraphics[width=0.9\textwidth]{js8.png}
	\caption{contact.js / fonctions d'erreurs}
	
	\label{fig:structure_main}
\end{figure}

\textbf{Comment ça marche :}
\begin{itemize}
	\item \texttt{setError()} : Affiche un message d'erreur sous un champ
	\item \texttt{clearError()} : Efface le message d'erreur
	\item \texttt{setStatus()} : Affiche un message global (succès ou erreur)
	\item \texttt{clearStatus()} : Efface le message global
\end{itemize}

\subsection{Effacement automatique des erreurs}

Pour améliorer l'expérience utilisateur, on efface les erreurs dès que l'utilisateur commence à corriger :

\begin{figure}[H] 
	\centering
	\includegraphics[width=0.9\textwidth]{js13.png}
	\caption{contact.js / fonctions d'erreurs}
	
	\label{fig:structure_main}
\end{figure}

On parcourt tous les champs et on ajoute un écouteur d'événement \texttt{input} qui efface l'erreur dès que l'utilisateur tape quelque chose.

\subsection{Compteur de caractères pour le message}

Le champ message est limité à 500 caractères. Nous affichons un compteur en temps réel :

\begin{figure}[H] 
	\centering
	\includegraphics[width=0.95\textwidth]{js12.png}
	\caption{contact.js /  Compteur de caracteres}
	
	\label{fig:structure_main}
\end{figure}

\textbf{Fonctionnement :}
\begin{enumerate}
	\item On compte le nombre de caractères avec \texttt{message.value.length}
	\item On affiche "X / 500"
	\item Si on dépasse 500, on affiche un avertissement
	\item Si on approche de 500 (entre 450 et 500), on ajoute une classe \texttt{warn} (couleur orange par exemple)
	\item Si on dépasse vraiment, on coupe le texte avec \texttt{slice(0, 500)}
\end{enumerate}

% [IMAGE: Compteur de caractères affichant "245 / 500"]

\subsection{Prévisualisation et validation du fichier}

Quand l'utilisateur choisit un fichier, on vérifie qu'il est valide et on affiche un aperçu :

\begin{figure}[H] 
	\centering
	\includegraphics[width=\textwidth]{js11.png}
	\caption{contact.js /Validation des fichiers}
	
	\label{fig:structure_main}
\end{figure}

\textbf{Explication étape par étape :}

\begin{enumerate}
	\item \textbf{Récupération du fichier} : \texttt{fileInput.files[0]} donne le premier fichier sélectionné
	
	\item \textbf{Extraction de l'extension} : 
	\begin{itemize}
		\item \texttt{name.split('.')} découpe le nom par les points
		\item \texttt{.pop()} prend la dernière partie (l'extension)
		\item \texttt{.toLowerCase()} convertit en minuscules
	\end{itemize}
	
	\item \textbf{Vérification de l'extension} : On vérifie que l'extension est dans notre liste autorisée
	
	\item \textbf{Vérification de la taille} : 5 MB = 5 × 1024 × 1024 octets
	
	\item \textbf{Prévisualisation} :
	\begin{itemize}
		\item Pour les images : on crée un \texttt{<img>} et on utilise \texttt{FileReader} pour charger l'image
		\item Pour les PDF : on affiche juste le nom du fichier
	\end{itemize}
\end{enumerate}

% [IMAGE: Previsualisation d'une image uploadee dans le formulaire]

\subsection{Validation de l'email}

Nous avons créé une fonction pour vérifier que l'email est bien formaté :

\begin{figure}[H] 
	\centering
	\includegraphics[width=0.85\textwidth]{js10.png}
	\caption{contact.js/ Validation d'email }
	
	\label{fig:structure_main}
\end{figure}

Cette fonction utilise une expression régulière (regex) pour vérifier le format :
\begin{itemize}
	\item \texttt{[\textbackslash w-.]+} : Au moins un caractère (lettre, chiffre, tiret, point)
	\item \texttt{@} : Le symbole arobase obligatoire
	\item \texttt{([\textbackslash w-]+\textbackslash .)+} : Nom de domaine avec au moins un point
	\item \texttt{[\textbackslash w-]\{2,\}} : Extension de domaine (au moins 2 caractères)
\end{itemize}

Exemples valides : \texttt{user@example.com}, \texttt{prenom.nom@domaine.fr}

Exemples invalides : \texttt{user@example}, \texttt{@example.com}, \texttt{user.example.com}

\subsection{Validation complète au moment de la soumission}

Quand l'utilisateur soumet le formulaire, on vérifie tous les champs :

\begin{figure}[H] 
	\centering
	\includegraphics[width=1.01\textwidth]{js9.png}
	\caption{contact.js/Validation Du Formulaire }
	
	\label{fig:structure_main}
\end{figure}

\textbf{Logique de validation :}

\begin{enumerate}
	\item \texttt{ev.preventDefault()} : Empêche l'envoi automatique du formulaire
	
	\item Variable \texttt{valid} : On part du principe que tout est valide (\texttt{true}), et on passe à \texttt{false} dès qu'on trouve une erreur
	
	\item \textbf{Validation du nom} : Au moins 2 caractères après suppression des espaces (\texttt{trim()})
	
	\item \textbf{Validation de l'âge} : 
	\begin{itemize}
		\item \texttt{parseInt(age.value, 10)} : Convertit le texte en nombre entier
		\item Vérifie que c'est un nombre valide et ≥ 18
	\end{itemize}
	
	\item \textbf{Validation de l'email} : Utilise notre fonction \texttt{isEmail()}
	
	\item \textbf{Validation du mot de passe} : Au moins 6 caractères
	
	\item \textbf{Confirmation du mot de passe} : Doit être identique au premier
	
	\item \textbf{Validation du fichier} : Même vérifications que lors du changement
	
	\item \textbf{Affichage du résultat} :
	\begin{itemize}
		\item Si erreur : message d'erreur en rouge
		\item Si succès : message de confirmation en vert
		\item \texttt{form.reset()} : Vide tous les champs
		\item \texttt{setTimeout()} : Efface le message après 5,5 secondes
	\end{itemize}
\end{enumerate}

% [IMAGE: Formulaire avec messages de validation (erreurs et succes)]


\subsection{Tests du menu mobile}

\begin{enumerate}
	\item Ouvrir le menu sur mobile → ✓ Fonctionne
	\item Fermer avec la touche Échap → ✓ Fonctionne
	\item Cliquer plusieurs fois sur le bouton → ✓ Ouvre/ferme correctement
\end{enumerate}

\subsection{Tests du formulaire}

\begin{table}[h]
	\centering
	\caption{Tests de validation du formulaire}
	\begin{tabular}{|p{5cm}|p{5cm}|p{4cm}|}
		\hline
		\textbf{Test} & \textbf{Résultat attendu} & \textbf{Résultat obtenu} \\
		\hline
		Nom vide & Message d'erreur & ✓ OK \\
		\hline
		Âge < 18 & Message d'erreur & ✓ OK \\
		\hline
		Email sans @ & Message d'erreur & ✓ OK \\
		\hline
		Mots de passe différents & Message d'erreur & ✓ OK \\
		\hline
		Fichier .exe & Message d'erreur & ✓ OK \\
		\hline
		Fichier > 5MB & Message d'erreur & ✓ OK \\
		\hline
		Tout valide & Message de succès & ✓ OK \\
		\hline
	\end{tabular}
\end{table}

\subsection{Outils de débogage utilisés}

\begin{itemize}
	\item \textbf{Console du navigateur} : Pour afficher les erreurs Javascript
	\item \textbf{Inspecteur d'éléments} : Pour vérifier les classes CSS ajoutées/retirées
	\item \textbf{Onglet Network} : Pour voir si les scripts se chargent correctement
	\item \textbf{Tests sur différents navigateurs} : Chrome, Firefox ,Edje
\end{itemize}

% [IMAGE: Console de developpement du navigateur]

\section{Conclusion du chapitre}

Dans ce chapitre, nous avons détaillé l'implémentation de Javascript dans notre projet ÉcoTourisme Maroc. Nous avons créé deux scripts principaux :

\begin{itemize}
	\item \textbf{main.js} : Gère les interactions générales (menu mobile, slider, navigation active, bouton retour en haut)
	\item \textbf{contact.js} : Gère la validation complète du formulaire de contact avec vérification en temps réel
\end{itemize}

Ces fonctionnalités Javascript améliorent considérablement l'expérience utilisateur en rendant le site plus interactif, plus réactif et plus agréable à utiliser. Bien que nous soyons débutants, nous avons réussi à implémenter des fonctionnalités importantes en suivant les bonnes pratiques du développement web moderne.

Le chapitre suivant présentera le déploiement du site en ligne sur GitHub Pages.


\chapter{Déploiement en ligne de la plateforme}

Le déploiement d'un site web consiste à le rendre accessible au public via Internet. Dans ce chapitre, nous expliquons comment nous avons mis en ligne le site ÉcoTourisme Maroc en utilisant GitHub Pages, une solution d'hébergement gratuite et simple pour les sites statiques.

\section{Qu'est-ce que le déploiement ?}

Le déploiement est l'étape finale du développement web qui permet de :
\begin{itemize}
	\item Rendre le site accessible à tout le monde via une URL
	\item Passer de l'environnement de développement (ordinateur local) à la production (serveur en ligne)
	\item Permettre aux utilisateurs de visiter le site depuis n'importe où dans le monde
\end{itemize}

\subsection{Différence entre développement local et production}

\begin{table}[h]
	\centering
	\caption{Développement local vs Production}
	\begin{tabular}{|p{5cm}|p{5cm}|}
		\hline
		\textbf{Développement local} & \textbf{Production (en ligne)} \\
		\hline
		Fichiers sur votre ordinateur & Fichiers sur un serveur \\
		\hline
		Accessible uniquement par vous & Accessible par tout le monde \\
		\hline
		URL : localhost ou file:/// & URL : https://site.com \\
		\hline
		Modifications instantanées & Nécessite redéploiement \\
		\hline
	\end{tabular}
\end{table}

% [IMAGE: Schema montrant ordinateur local vs serveur en ligne]

\section{Choix de GitHub Pages}

Pour héberger notre site, nous avons choisi GitHub Pages, une solution gratuite proposée par GitHub.

\subsection{Pourquoi GitHub Pages ?}

\textbf{Avantages :}
\begin{itemize}
	\item \textbf{Gratuit} : Hébergement illimité sans frais
	\item \textbf{Simple} : Déploiement automatique en quelques clics
	\item \textbf{HTTPS gratuit} : Certificat SSL automatique pour sécuriser le site
	\item \textbf{CDN intégré} : Le site se charge rapidement partout dans le monde
	\item \textbf{Pas de publicité} : Contrairement aux hébergeurs gratuits classiques
	\item \textbf{Intégration Git} : Mise à jour facile via Git
\end{itemize}

\textbf{Limitations :}
\begin{itemize}
	\item Uniquement pour les sites statiques (HTML, CSS, Javascript)
	\item Pas de base de données
	\item Pas de PHP ou autres langages serveur
\end{itemize}

Pour notre projet (site vitrine statique), GitHub Pages est parfaitement adapté.

\subsection{Alternatives considérées}

Nous avons comparé plusieurs solutions :

\begin{table}[h]
	\centering
	\caption{Comparaison des solutions d'hébergement}
	\begin{tabular}{|l|l|l|}
		\hline
		\textbf{Service} & \textbf{Prix} & \textbf{Complexité} \\
		\hline
		GitHub Pages & Gratuit & Facile \\
		\hline
		Netlify & Gratuit & Facile \\
		\hline
		Vercel & Gratuit & Moyenne \\
		\hline
		Hostinger & 2-10\euro/mois & Difficile \\
		\hline
	\end{tabular}
\end{table}

GitHub Pages a été retenu pour sa simplicité et son intégration native avec Git.

\section{Étapes du déploiement}

Le déploiement de notre site sur GitHub Pages s'est fait en plusieurs étapes simples.

\subsection{Étape 1 : Création du dépôt GitHub}

Nous avons créé un dépôt (repository) sur GitHub pour stocker notre code :

\begin{enumerate}
	\item Connexion à \texttt{github.com}
	\item Clic sur "New repository"
	\item Nom du dépôt : \texttt{ecotourismMaroc}
	\item Visibilité : Public (obligatoire pour GitHub Pages gratuit)
	\item Création du dépôt
\end{enumerate}

% [IMAGE: Capture d'ecran de la page de creation de depot GitHub]

\subsection{Étape 2 : Initialisation de Git en local}

Sur notre ordinateur, dans le dossier du projet :




\textbf{Explication des commandes :}
\begin{itemize}
	\item \texttt{git init} : Initialise un nouveau dépôt Git
	\item \texttt{git add .} : Ajoute tous les fichiers au suivi Git
	\item \texttt{git commit -m "..."} : Enregistre les modifications avec un message
	\item \texttt{git remote add origin ...} : Connecte le dépôt local à GitHub
	\item \texttt{git push} : Envoie le code sur GitHub
\end{itemize}

% [IMAGE: Terminal montrant les commandes Git executees]

\subsection{Étape 3 : Activation de GitHub Pages}

Dans les paramètres du dépôt sur GitHub :

\begin{enumerate}
	\item Aller dans \texttt{Settings} (paramètres du dépôt)
	\item Cliquer sur \texttt{Pages} dans le menu latéral
	\item Dans "Source", sélectionner :
	\begin{itemize}
		\item Branch : \texttt{main}
		\item Folder : \texttt{/ (root)}
	\end{itemize}
	\item Cliquer sur \texttt{Save}
	\item Attendre 1-2 minutes
\end{enumerate}

GitHub Pages génère automatiquement le site à l'adresse :
\begin{center}
	\texttt{https://mouad-arr.github.io/ecotourismMaroc/}
\end{center}

% [IMAGE: Parametres GitHub Pages avec la configuration]

\subsection{Étape 4 : Vérification du déploiement}

Après quelques minutes, nous avons :
\begin{enumerate}
	\item Ouvert l'URL du site dans le navigateur
	\item Vérifié que toutes les pages s'affichent correctement
	\item Testé les liens de navigation
	\item Vérifié que les images se chargent
	\item Testé le menu mobile
	\item Vérifié le formulaire de contact
\end{enumerate}

\textbf{Résultat :} Le site est en ligne et accessible publiquement ! ✓

% [IMAGE: Site web deploye visible dans le navigateur]

\section{Mise à jour du site}

Un des avantages de GitHub Pages est la facilité de mise à jour. Chaque fois que nous modifions le code et que nous le poussons sur GitHub, le site se met à jour automatiquement.

\subsection{Processus de mise à jour}

\begin{lstlisting}[language=bash, caption=Mise a jour du site]
	# 1. Faire des modifications dans le code
	
	# 2. Voir les fichiers modifies
	git status
	
	# 3. Ajouter les modifications
	git add .
	
	# 4. Commit avec un message descriptif
	git commit -m "Ajout: nouvelle destination Oasis du Sud"
	
	# 5. Envoyer sur GitHub
	git push origin main
	
	# 6. Attendre 1-2 minutes : le site se met a jour automatiquement
\end{lstlisting}

\textbf{Temps de déploiement :} Entre 30 secondes et 2 minutes après le push.

\subsection{Exemple de mise à jour}

Nous avons fait plusieurs mises à jour après le déploiement initial :
\begin{itemize}
	\item Correction de fautes d'orthographe
	\item Ajout de nouvelles images
	\item Amélioration du formulaire de contact
	\item Optimisation du menu mobile
\end{itemize}

Chaque mise à jour a suivi le même processus simple : modifier → commit → push.

\section{Sécurité : HTTPS automatique}

GitHub Pages active automatiquement HTTPS (protocole sécurisé) pour notre site. C'est très important pour plusieurs raisons :

\subsection{Qu'est-ce que HTTPS ?}

HTTPS (HyperText Transfer Protocol Secure) est la version sécurisée du HTTP. Il utilise le chiffrement SSL/TLS pour protéger les données.

\textbf{Avantages de HTTPS :}
\begin{itemize}
	\item \textbf{Sécurité} : Les données échangées sont chiffrées
	\item \textbf{Confiance} : Les navigateurs affichent un cadenas vert
	\item \textbf{SEO} : Google favorise les sites HTTPS dans les résultats de recherche
	\item \textbf{Moderne} : Standard actuel du web
\end{itemize}

% [IMAGE: Barre d'adresse du navigateur montrant le cadenas HTTPS]

\subsection{Configuration automatique}

Avec GitHub Pages :
\begin{enumerate}
	\item Le certificat SSL est généré automatiquement
	\item Il se renouvelle automatiquement
	\item Aucune configuration manuelle nécessaire
	\item C'est totalement gratuit
\end{enumerate}

Il suffit de cocher "Enforce HTTPS" dans les paramètres GitHub Pages (déjà activé par défaut).


\subsection{Vérification du code}

Nous avons vérifié que notre code HTML et CSS est correct :

\begin{itemize}
	\item \textbf{HTML} : Validation avec le W3C Validator (\texttt{validator.w3.org})
	\item \textbf{CSS} : Validation avec le CSS Validator
	\item \textbf{Javascript} : Vérification des erreurs dans la console du navigateur
\end{itemize}

Toutes les erreurs détectées ont été corrigées avant le déploiement final.

\subsection{Tests responsive}

Nous avons testé l'affichage du site sur différentes tailles d'écran :
\begin{itemize}
	\item Mobile (320px - 480px)
	\item Tablette (768px - 1024px)
	\item Desktop (> 1200px)
\end{itemize}

Tous les tests étaient positifs : le site s'affiche correctement partout.

\section{Conclusion du chapitre}

Le déploiement de notre site ÉcoTourisme Maroc sur GitHub Pages s'est déroulé avec succès. Nous avons réussi à :

\begin{itemize}
	\item Mettre le site en ligne gratuitement
	\item Obtenir une URL publique et un certificat HTTPS
	\item Établir un processus simple pour les mises à jour futures
	\item Optimiser le site pour de bonnes performances
\end{itemize}

Le site est maintenant accessible à l'adresse \texttt{https://mouad-arr.github.io/ecotourismMaroc/} et peut être consulté par n'importe qui dans le monde.

Cette expérience de déploiement nous a permis de comprendre :
\begin{itemize}
	\item Le fonctionnement de Git et GitHub
	\item La différence entre développement local et production
	\item L'importance de l'optimisation pour le web
	\item Le processus complet de mise en ligne d'un site web
\end{itemize}

\chapter*{Conclusion générale}
\addcontentsline{toc}{chapter}{Conclusion générale}

Le projet ÉcoTourisme Maroc nous a permis de découvrir et de pratiquer les technologies fondamentales du développement web moderne : HTML5, CSS3 et Javascript. À travers ce projet, nous avons créé un site web complet, du code initial au déploiement en ligne.

\section*{Objectifs atteints}

Nous avons réussi à créer un site web fonctionnel qui répond aux objectifs fixés :

\begin{itemize}
	\item \textbf{Site responsive} : Le site s'adapte correctement aux mobiles, tablettes et ordinateurs
	\item \textbf{Navigation intuitive} : Menu clair avec des liens fonctionnels vers toutes les pages
	\item \textbf{Design attractif} : Interface moderne avec des couleurs cohérentes et des images de qualité
	\item \textbf{Interactivité Javascript} : Menu mobile, slider, formulaire validé, bouton retour en haut
	\item \textbf{Déploiement réussi} : Site en ligne et accessible publiquement
\end{itemize}

\section*{Compétences acquises}

Ce projet nous a permis d'apprendre et de pratiquer de nombreuses compétences techniques :

\subsection*{HTML}
\begin{itemize}
	\item Structure sémantique des pages
	\item Formulaires et validation
	\item Organisation du contenu
	\item Balises meta et SEO de base
\end{itemize}

\subsection*{CSS}
\begin{itemize}
	\item Flexbox et CSS Grid pour les layouts
	\item Media Queries pour le responsive design
	\item Animations et transitions
	\item Variables CSS pour la cohérence
	\item Gestion des couleurs et typographie
\end{itemize}

\subsection*{Javascript}
\begin{itemize}
	\item Manipulation du DOM
	\item Gestion des événements
	\item Validation de formulaires
	\item Création de fonctionnalités interactives
	\item Bonnes pratiques de code
\end{itemize}

\subsection*{Git et déploiement}
\begin{itemize}
	\item Versioning du code avec Git
	\item Utilisation de GitHub
	\item Déploiement avec GitHub Pages
	\item Workflow de mise à jour
\end{itemize}

\section*{Difficultés rencontrées et solutions}

En tant que débutants, nous avons rencontré plusieurs défis :

\subsection*{Responsive Design}
\textbf{Difficulté :} Adapter le site aux différentes tailles d'écran

\textbf{Solution :} Utilisation de Flexbox, CSS Grid et Media Queries après étude de tutoriels et exemples

\subsection*{Javascript}
\textbf{Difficulté :} Comprendre la logique de programmation et la manipulation du DOM

\textbf{Solution :} Découpage en petites fonctions, tests fréquents dans la console, documentation MDN

\subsection*{Git}
\textbf{Difficulté :} Comprendre les concepts de commit, push, branches

\textbf{Solution :} Apprentissage progressif des commandes de base, aide du professeur

\section*{Points forts du projet}

\begin{itemize}
	\item \textbf{Fonctionnalité complète} : Le site contient toutes les pages prévues et toutes les fonctionnalités fonctionnent
	\item \textbf{Design cohérent} : Charte graphique respectée sur toutes les pages
	\item \textbf{Accessibilité} : Navigation au clavier, attributs ARIA, messages d'erreur clairs
	\item \textbf{Performance} : Images optimisées, code validé
	\item \textbf{En ligne} : Site déployé et accessible publiquement
\end{itemize}

\section*{Améliorations futures possibles}

Si nous devions continuer à développer ce site, nous pourrions ajouter :

\subsection*{Court terme}
\begin{itemize}
	\item Plus de destinations et d'activités
	\item Galerie photos avec lightbox
	\item Système de recherche de destinations
	\item Blog avec articles sur l'écotourisme
	\item Page FAQ (questions fréquentes)
\end{itemize}

\subsection*{Moyen terme}
\begin{itemize}
	\item Backend avec base de données
	\item Système de réservation en ligne
	\item Comptes utilisateurs
	\item Système de commentaires et avis
	\item Newsletter
\end{itemize}

\subsection*{Long terme}
\begin{itemize}
	\item Application mobile
	\item Carte interactive avec géolocalisation
	\item Système de recommandations personnalisées
	\item Partenariats avec éco-lodges
	\item Intégration de paiement en ligne
\end{itemize}

\section*{Impact du projet}

Au-delà de l'apprentissage technique, ce projet a un objectif de sensibilisation au tourisme durable au Maroc. Notre site vise à :

\begin{itemize}
	\item Promouvoir les destinations naturelles marocaines
	\item Encourager des pratiques de voyage responsables
	\item Valoriser le patrimoine culturel et environnemental
	\item Soutenir les communautés locales
\end{itemize}

\section*{Remerciements}

Nous tenons à remercier :

\begin{itemize}
	\item Le Professeur Qazdar Aimad pour son encadrement et ses conseils
	\item L'ENSA pour la formation en technologies web
	\item Nos camarades pour les échanges et l'entraide
\end{itemize}

\section*{Conclusion finale}

Ce projet a été une expérience d'apprentissage très enrichissante. Partir d'une page blanche et arriver à un site web complet et déployé en ligne nous a donné confiance en nos capacités de développement web.

Nous avons découvert que créer un site web demande de la rigueur, de la patience et de la créativité. Chaque problème rencontré nous a appris quelque chose de nouveau. Le résultat final, bien qu'il soit celui de débutants, est fonctionnel et nous en sommes fiers.

Cette expérience nous a donné envie de continuer à apprendre le développement web et à améliorer nos compétences. Le site ÉcoTourisme Maroc est maintenant en ligne à l'adresse :

\begin{center}
	\texttt{https://mouad-arr.github.io/ecotourismMaroc/}
\end{center}

C'est le début d'une aventure dans le monde du développement web, et nous sommes motivés pour continuer à apprendre et à créer.

\end{document}